\documentclass{lmcs}
\pdfoutput=1

\usepackage{enumerate}
\usepackage[colorlinks=true]{hyperref}
\usepackage{amssymb}
\usepackage{xcolor,latexsym,amsmath,extarrows,alltt}
\usepackage{xspace}
\usepackage{booktabs}
\usepackage{mathtools}
\usepackage{enumitem}
\usepackage{stmaryrd}
\usepackage{microtype}

\theoremstyle{theorem}\newtheorem{theorem}{Theorem}
\theoremstyle{theorem}\newtheorem{lemma}[theorem]{Lemma}
\theoremstyle{theorem}\newtheorem{corollary}[theorem]{Corollary}
\theoremstyle{definition}\newtheorem{definition}[theorem]{Definition}
\theoremstyle{definition}\newtheorem{example}[theorem]{Example}

\newcommand{\N}{\mathbb{N}}
\newcommand{\F}{\mathcal{F}}
\newcommand{\V}{\mathcal{V}}
\newcommand{\Vfree}{\mathcal{V}_{\mathit{free}}}
\newcommand{\Vbound}{\mathcal{V}_{\mathit{bound}}}
\newcommand{\Sorts}{\mathcal{S}}
\newcommand{\Types}{\mathcal{Y}}
\newcommand{\Terms}{\mathcal{T}}
\newcommand{\ATerms}{\mathcal{T}_{\mathcal{A}}}
\newcommand{\FOTerms}{\mathcal{T}_{\mathcal{FO}}}
\newcommand{\Rules}{\mathcal{R}}
\newcommand{\FV}{\mathit{FV}}
\newcommand{\Positions}{\mathit{Positions}}
\newcommand{\Pairs}{\mathit{Pairs}}

\newcommand{\domain}{\mathtt{dom}}
\newcommand{\order}{\mathit{order}}

\newcommand{\asort}{\iota}
\newcommand{\bsort}{\kappa}
\newcommand{\atype}{\sigma}
\newcommand{\btype}{\tau}
\newcommand{\ctype}{\pi}
\newcommand{\dtype}{\alpha}
\newcommand{\identifier}[1]{\mathtt{#1}}
\newcommand{\afun}{\identifier{f}}
\newcommand{\bfun}{\identifier{g}}
\newcommand{\cfun}{\identifier{h}}
\newcommand{\avar}{x}
\newcommand{\bvar}{y}
\newcommand{\cvar}{z}
\newcommand{\Avar}{X}
\newcommand{\Bvar}{Y}
\newcommand{\Cvar}{Z}
\newcommand{\AFvar}{F}
\newcommand{\BFvar}{G}
\newcommand{\CFvar}{H}

\newcommand{\abs}[2]{\lambda #1.#2}

\newcommand{\arity}{\mathit{arity}}
\newcommand{\head}{\mathsf{head}}
\newcommand{\arrtype}{\rightarrow}
\newcommand{\arrz}{\Rightarrow}
\newcommand{\arr}[1]{\arrz_{#1}}
\newcommand{\arrr}[1]{\arr{#1}^*}
\newcommand{\subtermeq}{\unlhd}
\newcommand{\headsubtermeq}{\unlhd_{\bullet}}
\newcommand{\supterm}{\rhd}
\newcommand{\suptermeq}{\unrhd}

\newcommand{\symb}[1]{\mathtt{#1}}

\newcommand{\nul}{\symb{0}}
\newcommand{\one}{\symb{1}}
\newcommand{\nil}{\symb{nil}}
\newcommand{\cons}{\symb{cons}}
\newcommand{\strue}{\symb{true}}
\newcommand{\sfalse}{\symb{false}}
\newcommand{\suc}{\symb{s}}
\newcommand{\map}{\symb{map}}
\newcommand{\bool}{\symb{bool}}
\newcommand{\nat}{\symb{nat}}
\newcommand{\lijst}{\symb{list}}
\newcommand{\unitsort}{\mathtt{o}}

\newcommand{\cora}{\textsf{CORA}\xspace}

\newcommand{\secshort}{\S}
\newcommand{\myparagraph}[1]{\paragraph{\textbf{#1}}}

\setlength{\parindent}{0pt}
\setlength{\parskip}{\bigskipamount}
\setlist[itemize]{topsep=-\bigskipamount}

\newcommand{\mysubsection}[1]{\vspace{-12pt}\subsubsection{#1}}

\begin{document}

\title{COnstrained Rewriting Analyser: formalism}
\author{Cynthia Kop}
\address{Department of Software Science, Radboud University Nijmegen}
\email{C.Kop@cs.ru.nl}

\maketitle

\begin{abstract}
\cora\ is a tool meant to analyse constrained term rewriting systems, both
first-order and higher-order.  This document explains the underlying formalism.
\end{abstract}

\section{Types}

We fix a set $\Sorts$ of \emph{sorts} and define the set $\Types$ of \emph{types} inductively:
\begin{itemize}
\item all elements of $\Sorts$ are types (also called \emph{base types});
\item if $\atype,\btype \in \Types$ then $\atype \arrtype \btype$ is also a type (called an arrow
  type).
\end{itemize}
The arrow operator $\arrtype$ is right-associative, so all types can be denoted in a form
$\atype_1 \arrtype \dots \arrtype \atype_m \arrtype \asort$ with $\asort \in \Sorts$; we say the
\emph{arity} of this type is $m$, and the \emph{output sort} is $\asort$.

The \emph{order} of a type is recursively defined as follows:
\begin{itemize}
\item for $\asort \in \Sorts$: $\order(\asort) = 0$;
\item for arrow types: $\order(\atype \arrtype \btype) = \max(\order(\atype) + 1,\order(\btype))$.
\end{itemize}

\bigskip
Type equality is literal equality (i.e., $\atype_1 \arrtype \btype_1$ is equal to $\atype_2 \arrtype
\btype_2$ iff $\atype_1 = \atype_2$ and $\btype_1 = \btype_2$).

\subsection*{Remarks}

We do not impose limitations on the set of sorts.  In traditional, unsorted term rewriting, there
is only one sort (e.g., $\Sorts = \{ \unitsort \}$). However, we may also have a larger finite or
even infinite sort set.
In the future, we may consider a shallow form of polymorphic types, but for the moment we will limit
interest to these simple types.

\section{Unconstrained higher-order term rewriting systems (HOTRSs)}

Let us start by explaining systems without constraints. Most of the notions will be directly
relevant to constrained systems as well.

\subsection{Terms}
Terms are \emph{well-typed} expressions built over given sets of \emph{function symbols} and
\emph{variables}. The full definition is presented below.

\mysubsection{Symbols and variables}

We fix a set $\F$ of \emph{function symbols}, also called the \emph{alphabet}; each function symbol
is a \emph{typed constant}. Notation: $\afun \in \F$ or $(\afun : \atype) \in \F$ if we wish to
explicitly refer to the type (but the type should be considered implicit in the symbol).
Function symbols will generally be referred to as $\afun,\bfun,\cfun$ or using more suggestive
notation.

We also fix a set $\V$ of \emph{variables}, which are typed constants in the same way.  $\V$ should
be disjoint from $\F$, and we assume that $V = \Vfree \uplus \Vbound$, where $\Vbound$ contains
infinitely many variables of each type; there are no restrictions on $\Vfree$.
Variables will generally be referred to as $\avar,\bvar,\cvar,\Avar,\Bvar,\Cvar,\AFvar,\BFvar,
\CFvar$ or using more suggestive notation.

\mysubsection{Term formation}

Terms are those expressions $s$ such that $s : \atype$ can be derived for some $\atype \in \Types$
using the following clauses:
\begin{itemize}
\item if $(\afun : \atype_1 \arrtype \dots \arrtype \atype_n \arrtype \btype) \in \F$ and
  $s_1 : \atype_1,\dots,s_n : \atype_n$ then $\afun(s_1,\dots,s_n) : \btype$;
\item if $(\avar : \atype_1 \arrtype \dots \arrtype \atype_n \arrtype \btype) \in \V$ and
  $s_1 : \atype_1,\dots,s_n : \atype_n$ then $\avar(s_1,\dots,s_n) : \btype$;
\item if $(\avar : \atype) \in \Vbound$ and $s : \btype$ then $\abs{\avar}{s} : \atype \arrtype
  \btype$.
\end{itemize}
A term of the form $\afun(s_1,\dots,s_n)$ is called a \emph{functional term} and $\afun$ is its
root.
A term of the form $\avar(s_1,\dots,s_n)$ is called a \emph{var term}, and $\avar$ is its variable.
A term of the form $\abs{\avar}{s}$ is called an \emph{abstraction} and $\avar$ is its variable.
If $s : \atype$ then we say that $\atype$ is the type of $s$; it is clear from the definitions
above that each term has a unique type.

Note that in the definition above, $n$ is not required to be maximal; for example, if
$\symb{greater} : \mathtt{int} \arrtype \mathtt{int} \arrtype \mathtt{bool}$, then each of
$\symb{greater}(),\symb{greater}(\avar)$ and $\symb{greater}(\avar,\bvar)$ are terms (with
distinct types). When no arguments are given (i.e., a term $\afun()$ or $\avar()$), we may omit
the brackets and just denote $\afun$ or $\avar$ for the term.  In this sense, the elements of $\F$
and $\V$ may be considered terms.
A term $\avar$ can simply be called a variable (but is also still a var term);
a term $\afun$ may be called a constant (but is also still a functional term).

\mysubsection{$\alpha$-equality}
We let $=_\alpha$ be the usual $\alpha$-renaming equivalence relation as used in the
$\lambda$-calculus. This relation can be formally defined as follows:
\begin{itemize}
\item Let $\mu_0,\nu_0 : \V \rightarrow \N$ be defined as follows:
  $\mu_0(\avar) = 0$ and $\nu_0(\avar) = 0$ for all $\avar \in \V$.
\item Let $s =_\alpha t$ iff $s =_\alpha^{\mu_0,\nu_0,1} t$.
\item For $\mu,\nu : \V \rightarrow \N$ and $k \in \N$, let $=_\alpha^{\mu,\nu,k}$ be defined as
  follows:
  \begin{itemize}
  \item $\afun(s_1,\dots,s_n) =_\alpha^{\mu,\nu,k} t$ iff $t = \afun(t_1,\dots,t_n)$ with $s_1
    =_\alpha^{\mu,\nu,k} t_1,\dots,s_n =_\alpha^{\mu,\nu,k} t_n$;
  \item $\avar(s_1,\dots,s_n) =_\alpha^{\mu,\nu,k} t$ iff:
    \begin{itemize}
    \item $t = \bvar(t_1,\dots,t_n)$ and
    \item $s_1 =_\alpha^{\mu,\nu,k} t_1,\dots,s_n =_\alpha^{\mu,\nu,k} t_n$, and
    \item either $\avar = \bvar$ and $\mu(\avar) = \nu(\avar) = 0$,
      or $\mu(\avar) = \nu(\avar) > 0$.
    \end{itemize}
  \item $\abs{\avar}{s} =_\alpha^{\mu,\nu,k} t$ iff
    $t = \abs{\bvar}{u}$ and $s =_\alpha^{\mu[\avar:=k],\nu[\bvar:=k],k+1} u$. \\
    (Here, $\mu[\avar:=k]$ is the function that maps $\avar$ to $k$ and all other $\cvar$ to
    $\mu(\cvar)$; similar for $\nu[\bvar:=k]$.)
  \end{itemize}
\end{itemize}
That is, we progressively descend into the term and keep track of where variables are bound; the
structure of the two terms has to be exactly the same, and function symbols and unbound variable
should occur at the same positions in both terms. However, when encountering a bound variable, we
only require that this variable was bound by the same $\lambda$ in both terms.

\mysubsection{Sets of terms}

We denote $\Terms(\F,\V)$ for the set of all terms $s$, modulo $=_\alpha$.  In practice, we will
reason with terms rather than equivalence classes, but always consider equality modulo $=_\alpha$.

A term $s$ is a \emph{pattern} if for every subterm $t \subtermeq s$ we have: if $t$ is a var term
$x(s_1,\dots,s_n)$ with $x \in \Vfree$, then $s_1,\dots,s_n$ are distinct elements of $\Vbound$.
Patterns will be relevant in rule formation for specific limitations of HOTRSs.

A term $s$ is \emph{applicative} if it does not use variables in $\Vbound$ (and therefore, no
abstractions are constructed); that is, a term is applicative if it can be typed by the
following clauses:
\begin{itemize}
\item if $(\afun : \atype_1 \arrtype \dots \arrtype \atype_n \arrtype \btype) \in \F$ and
  $s_1 : \atype_1,\dots,s_n : \atype_n$ then $\afun(s_1,\dots,s_n) : \btype$;
\item if $(\avar : \atype_1 \arrtype \dots \arrtype \atype_n \arrtype \btype) \in \Vfree$ and
  $s_1 : \atype_1,\dots,s_n : \atype_n$ then $\avar(s_1,\dots,s_n) : \btype$.
\end{itemize}
The set of applicative terms is denoted $\ATerms(\F,\V)$.  Note that $=_\alpha$ is the
identity on applicative terms, so $\ATerms(\F,\V) \subsetneq \Terms(\F,\V)$.
Note also that, in an applicative \emph{pattern}, variables are not allowed to occur at the head
at all: every var term is a variable.

A term is \emph{first-order} if its type can be derived by the following clauses:
\begin{itemize}
\item if $(\afun : \atype_1 \arrtype \dots \arrtype \atype_n \arrtype \asort) \in \F$ with $\asort
  \in \Sorts$ and
  $s_1 : \atype_1,\dots,s_n : \atype_n$ then $\afun(s_1,\dots,s_n) : \asort$;
\item if $(\avar : \asort) \in \Vfree$ and $\asort \in \Sorts$, then $\avar : \asort$.
\end{itemize}
The set of first-order terms is denoted $\FOTerms(\F,\V)$.  Note that every first-order term is
also an applicative term (indeed -- every first-order term is an applicative \emph{pattern}),
so $\FOTerms(\F,\V) \subseteq \ATerms(\F,\V) \subsetneq \Terms(\F,\V)$.

\subsection{Free variables}
The set of \emph{free variables} of a term is inductively defined as follows:
\begin{itemize}
\item $\FV(\afun(s_1,\dots,s_n)) = \FV(s_1) \cup \dots \cup \FV(s_n)$;
\item $\FV(\avar(s_1,\dots,s_n)) = \{ \avar \} \cup \FV(s_1) \cup \dots \cup \FV(s_n)$;
\item $\FV(\abs{\avar}{s}) = \FV(s) \setminus \{ \avar \}$.
\end{itemize}
That is, $\FV(s)$ contains all variables in $s$ except for those bound by a $\lambda$.
For applicative and first-order terms $s$, this is the set of \emph{all} variables occurring in
$s$.

\subsection{Subterms and positions}

The \emph{positions} of a given term are the paths to specific subterms, defined as follows:

\begin{itemize}
\item $\Positions(\afun(s_1,\dots,s_n)) = \{ \epsilon \} \cup \{ i \cdot p \mid 1 \leq i
  \leq n \wedge p \in \mathit{Positions}(s_i) \}$;
\item $\Positions(\avar(s_1,\dots,s_n)) = \{ \epsilon \} \cup \{ i \cdot p \mid 1 \leq i
  \leq n \wedge p \in \Positions(s_i) \}$;
\item $\Positions(\abs{\avar}{s}) = \{ 0 \cdot p \mid p \in \Positions(s) \}$.
\end{itemize}
Note that positions are associated to a term; thus, not every sequence of natural numbers is a
position.

For a term $s$ and a position $p \in \Positions(s)$, the \emph{subterm of $s$ at position $p$},
denoted $s|_p$, is defined as follows:
\begin{itemize}
\item $s|_\epsilon = s$;
\item $\afun(s_1,\dots,s_n)|_{i \cdot p} = s_i|_p$;
\item $\avar(s_1,\dots,s_n)|_{i \cdot p} = s_i|_p$;
\item $(\abs{\avar}{s})|_{0 \cdot p} = s|_p$.
\end{itemize}

If $s|_p$ has the same type as some term $t$, then $s[t]_p$ denotes $s$ with the subterm at position
$p$ replaced by $t$.  Formally, $s[t]_p$ is obtained as follows:
\begin{itemize}
\item $s[t]_p = t$;
\item $\afun(s_1,\dots,s_n)[t]_{i \cdot p} = \afun(s_1,\dots,s_{i-1},s_i[t]_p,s_{i+1},\dots,s_n)$;
\item $\avar(s_1,\dots,s_n)[t]_{i \cdot p} = \avar(s_1,\dots,s_{i-1},s_i[t]_p,s_{i+1},\dots,s_n)$.
\end{itemize}
Thus, we can find and replace the subterm at a given position.

We say that \emph{$t$ is a subterm of $s$}, notation $t \subtermeq s$, if there is some position
$p \in \Positions(s)$ with $t = s|_p$.  This could equivalently be formulated as follows:

\begin{lemma}
$t \subtermeq s$ if and only if one of the following holds:
\begin{itemize}
\item $s = t$;
\item $s = \afun(s_1,\dots,s_n)$ or $s = \avar(s_1,\dots,s_n)$ and $t \subtermeq s_i$ for some $i$;
\item $s = \abs{x}{s'}$ and $t \subtermeq s'$.
\end{itemize}
\end{lemma}

It should be noted that in contrast to most definitions of higher-order rewriting, we do \emph{not}
consider, for example, $\afun(x)$ to be a subterm of $\afun(x,y)$.  Instead, we define the
following: \emph{$t$ is a head-subterm of $s$}, notation $t \headsubtermeq s$ if one of the
following holds:
\begin{itemize}
\item $t \subtermeq s$;
\item $t = \afun(s_1,\dots,s_i)$ and there exist $s_{i+1},\dots,s_n$ such that
  $\afun(s_1,\dots,s_n) \subtermeq s$;
\item $t = \avar(s_1,\dots,s_i)$ and there exist $s_{i+1},\dots,s_n$ such that
  $\avar(s_1,\dots,s_n) \subtermeq s$;
\end{itemize}
So, the head-subterms of $s$ are both the subterms of $s$, and those terms that occur as the head
of a subterm of $s$.

It should also be noted that if $s =_\alpha t$, it does not follow that $s$ and $t$ have the same
subterms: $\abs{x}{x}$ has a subterm $x$, while $\abs{y}{y}$ does not.  For applicative and
first-order terms, this is not an issue.

Regarding different kinds of terms: the subterms and positions of a first-order term by these
definitions are exactly the subterms and positions as they are usually considered in first-order
term rewriting; however, head-subterms are generally not considered.  For applicative terms,
both subterms and head-subterms are usually referred to as just ``subterms''; we distinguish them
here because doing so is practical for analysis.

\subsection{Application and substitution}

A \emph{substitution} is a function $\gamma$ that maps each variable $\avar \in \V$ to a term
$\gamma(\avar)$ of the same type.
The \emph{domain} $\domain(\gamma)$ of a substitution $\gamma$ is the set of all variables $x$
such that $\gamma(x) \neq x$.
We denote $[x_1:=s_1,\dots,x_n:=s_n]$ for the substitution $\gamma$ with $\gamma(x_i) = s_i$ for
$1 \leq i \leq n$ and $\gamma(y) = y$ for $y \notin \{x_1,\dots,x_n\}$.

Applying a substitution $\gamma$ to a term $s$, notation $s\gamma$, yields a new term of the same
type, as we will define below. However, this requires a separate definition of \emph{application}.
We will define the notions of term application and substitution in a mutually recursive manner.

\mysubsection{Term application}\label{mysubsec:application}
A term $s : \atype_1 \arrtype \dots \arrtype \atype_m \arrtype \btype$ can be applied to a sequence
$[t_1,\dots,t_m]$ of terms, provided $t_1 : \atype_1,\dots,t_m : \atype_m$, through the following
clauses:
\begin{itemize}
\item $s \cdot [t_1,\dots,t_m] = s$ if $m = 0$;
\item $s \cdot [t_1,\dots,t_m] = (s \cdot t_1) \cdot [t_2,\dots,t_m]$ otherwise;
\item if $s = \afun(s_1,\dots,s_n)$ then $s \cdot t = \afun(s_1,\dots,s_n,t)$;
\item if $s = \avar(s_1,\dots,s_n)$ then $s \cdot t = \avar(s_1,\dots,s_n,t)$;
\item if $s = \abs{\avar}{s}$ then $s \cdot t = s[\avar:=t]$ (using substitution, 
  see section \ref{mysubsec:substitution}).
\end{itemize}

\bigskip
Note that for applicative terms, this definition is complete as substitution is only needed for
abstractions.  We have the following result;

\begin{lemma}\label{lem:applicative_notation}
The set $\ATerms(\F,\V)$ is the smallest set such that:
\begin{itemize}
\item $\F \cup \V \subseteq \ATerms(\F,\V)$;
\item if $s,t \in \ATerms(\F,\V)$ and $s : \atype \arrtype \btype$ and $t : \atype$ then
  $s \cdot t \in \ATerms(\F,\V)$.
\end{itemize}
\end{lemma}

\begin{proof}
Trivial.
\end{proof}

Lemma~\ref{lem:applicative_notation} shows that our applicative terms are the same as applicative
terms constructed in the traditional way; however, for convenience we denote them in a functional
notation.

\mysubsection{Substitution}\label{mysubsec:substitution}
A substitution $\gamma$ can be applied to a term as follows:
\begin{itemize}
\item $\afun(s_1,\dots,s_n)\gamma = \afun(s_1\gamma,\dots,s_n\gamma)$;
\item $\avar(s_1,\dots,s_n)\gamma = \gamma(\avar) \cdot [s_1\gamma,\dots,s_n\gamma]$
  (using application, see section \ref{mysubsec:application});
\item $(\abs{\avar}{s})\gamma = (\abs{\cvar}{s}) ([\avar:=\cvar] \cup [\bvar := \gamma(\bvar) \mid
  \bvar \in \domain(\gamma) \setminus \{\avar\}])$ for $\cvar$ a \emph{fresh}** variable in
  $\Vbound$ with the same type as $\avar$.
\end{itemize}
** A \emph{fresh} variable $\cvar$ is one that does not occur in $\FV(\gamma(\bvar))$ for any
$\bvar \in \FV(s)$.

In truth, this definition does not define a function on terms: since the substitution of an
abstraction may lead to any fresh variable being chosen.  Thus, applying a substitution to a term
can be seen as a relation.  However, as we will see in Lemma \ref{lem:substitutionalpha}: if
$s\gamma = t_1$ and $s\gamma = t_2$, then $t_1 =_\alpha t_2$.  Thus, this difference is not
significant, and substitutions do define a function on equivalence classes of terms.

For two substitutions $\gamma$ and $\delta$, we let $\gamma\delta$ denote the substitution that
maps each variable $x$ to $\gamma(x)\delta$.

\mysubsection{Well-definedness results}
To start, due to the mutual recursion it is non-trivial that application and substitution are
well-defined.  This critical result is given by Lemma \ref{lem:substdefined}.

\begin{lemma}\label{lem:substdefined}
The following results hold:
\begin{itemize}
\item For every term $s$ and substitution $\gamma$ there exists $t$ with $s\gamma = t$.
\item For every term $s : \atype_1 \arrtype \dots \atype_m \arrtype \btype$ and terms
  $t_1 : \atype_1,\dots,t_m : \atype_m$: there exists $t$ with $s \cdot [t_1,\dots,t_m] = t$.
\end{itemize}
\end{lemma}

\begin{proof}
Let $K_{s,\gamma} = 1 + 2 * \max\{ \order(\atype) \mid \bvar \in \FV(s) \cap \gamma(\bvar) \notin
\V \}$, and $L_s = 2 * \order(\atype)$ if $s : \atype$.  We prove the results by mutual induction
on $K_{s,\gamma}$ or $L_s$ first, $m$ second (for the second statement) and the size of $s$ third.

For the first statement:
\begin{itemize}
\item If $s = \afun(s_1,\dots,s_n)$, the result follows by the third induction hypothesis.
\item If $s = \abs{\avar}{s'}$ this is also the case, since $\{ \bvar \in \FV(s) \wedge
  \gamma(\bvar) \notin \V \} \subseteq \{ \bvar \in \FV(s) \wedge \delta(\bvar) \notin \V
  \}$ when $\delta(\avar)$ is a variable and $\delta(\bvar) = \gamma(\bvar)$ for all other
  variables.
\item If $s = \avar(s_1,\dots,s_n)$, then $s_1\gamma,\dots,s_n\gamma$ are all well-defined by
  the third induction hypothesis, which suffices if $\gamma(\avar) \in \V$ (since
  $s\gamma = \bvar(s_1\gamma,\dots,s_n\gamma)$ if $\gamma(\avar) = \bvar)$).
  Otherwise, let $\avar : \atype$ and $k : \order(\atype)$.  By definition, $K_{s,\gamma} \geq 1 +
  2 * k > 2 * k = L_{\gamma(\avar)}$.  Thus, $\gamma(\avar) \cdot [s_1\gamma,\dots,s_n\gamma]$
  is well-defined by the induction hypothesis.
\end{itemize}
For the second statement:
\begin{itemize}
\item If $s = \afun(s_1,\dots,s_n)$ then $s \cdot [t_1,\dots,t_m] = \afun(s_1,\dots,s_n,t_1,\dots,
  t_m)$.
\item Similar if $s = \avar(s_1,\dots,s_n)$.
\item If $s = \abs{\avar}{s'}$, $s : \atype \arrtype \btype$. We are done if $m = 0$, and if
  $m > 1$ we note that $\order(\atype \arrtype \btype) \geq \order(\btype)$, so $s \cdot [t_1]$ is
  defined by the second induction hypothesis and $(s \cdot [t_1]) \cdot [t_2,\dots,t_m]$ by the
  second or third.  The case remains where $m = 1$, so $s \cdot [t_1] = s'[\avar:=t]$.  However,
  then $\avar$ has type $\atype$, and $\order(\atype) < \order(\atype \arrtype \btype)$.  We
  complete with the first induction hypothesis.
  \qedhere
\end{itemize}
\end{proof}

So, substitution and application are well-defined.  We now note that they also define a function on
$=_\alpha$-equivalence classes:

\begin{lemma}\label{lem:substitutionalpha}
If $s =_\alpha s'$ and $s\gamma = t$ and $s'\gamma = t'$, then $t =_\alpha t'$.
Therefore, substitution defines a function on $\Terms(\F,\V)$.
\end{lemma}

\begin{proof}
Not hard, but tedious (omitted).
\end{proof}

\subsection{Rules and rewriting}

A rule $\rho$ is a pair $\ell \arrz r$ of two terms with the same type.

For a given set $T \subseteq \Terms(\F,\V)$, a rule generates the set $\mathit{Pairs}_{\rho,T} :=
\{ (u,v) \mid u,v \in T\ \wedge$ there exists a substitution $\gamma : \Vfree \to T$ and
$w_1,\dots,w_n \in T$ ($n \geq 0$) such that $u = (\ell\gamma) \cdot [w_1,\dots,w_n]$ and $v =
(r\gamma) \cdot [w_1,\dots,w_n] \}$.

For a given set of rules $\Rules$ and $T \subseteq \Terms(\F,\V)$, the reduction relation
$\arr{\Rules,T}$ is given by:
\begin{itemize}
\item if there is some $\rho \in \Rules$ and a pair $(u,v) \in \mathit{Pairs}_{\rho,T}$ such that
  $s|_p = u$, then $s \arr{\Rules} s[v]_p$.
\end{itemize}

\medskip
Note that this explicitly includes applications of rules at the head of a subterm.
We will denote $\arr{\Rules}$ for the relation $\arr{\Rules,\Terms(\F,\V)}$.

A rule is a \emph{pattern rule} if the left-hand side $\ell$ is a pattern.

\subsection{HOTRSs}

We now have all the ingredients to define a \emph{higher-order term rewriting system (HOTRS)}.

\mysubsection{Abstract Rewriting Systems}

An abstract rewriting system is a pair $(\mathcal{A},\arrz)$ where $\mathcal{A}$ is a set and
$\arrz$ a binary relation on that set.  Properties such as termination and confluence can be
expressed in terms of abstract rewriting systems.

\mysubsection{HOTRSs}

A higher-order term rewriting system (HOTRS) is an abstract rewriting system of the form
$(T,\arr{Rules,T})$ where $T \subseteq \Terms(\F,\V)$ and for all rules $\ell \arrz r \in \Rules$:
both $\ell$ and $r$ are in $T$.

\mysubsection{Examples of HOTRSs}
Many standard forms of term rewriting systems can be expressed as HOTRSs.

A \emph{many-sorted term rewriting system} (MTRS) is a HOTRS with $T = \FOTerms(\F,\V)$ and
$\F,\Rules$ with the following properties:
\begin{itemize}
\item for all $(\afun : \atype) \in \F$: $\order(\atype) \leq 1$;
\item for all $\ell \arrz r \in \Rules$: $\ell$ is not a variable, and $\FV(r) \subseteq \FV(\ell)$.
\end{itemize}
Moreover, in a many-sorted term rewriting system, the reduction relation is exactly $\arr{\Rules}$
(so we do not need to consider $\arr{\Rules,\FOTerms(\F,\V)}$): this is the case because, for
$s \in \FOTerms(\F,V)$ and $\ell \arrz r$ a rule with $\ell,r \in \FOTerms$, we have
$s \arr{\Rules} t$ if and only if there exist $p \in \Positions(s)$ and a substitution $\gamma$
mapping $\Vfree$ to $\FOTerms(\F,V)$ such that $s|_p = \ell\gamma$ and $t = s[r\gamma]_p$.

An \emph{unsorted first-order term rewriting system} (TRS) is a many-sorted term rewriting system
with $\Sorts = \{ \unitsort \}$.

An \emph{applicative term rewriting system} (ATRS) is a HOTRS with $T = \ATerms(\F,\V)$.  Here,
there actually \emph{is} a difference between $\arr{\Rules}$ and $\arr{\Rules,T}$, because in the
former case, a rule $\afun(\avar(\nul)) \arrz \afun(\avar(\symb{1}))$ would reduce
$\afun(\symb{2})$ to itself (through the substitution $\gamma = [\avar:=\abs{\bvar}{\symb{2}}]$),
while in the latter case this would not happen.  However, if all rules in $\Rules$ are
\emph{pattern rules}, then here too $\arr{\Rules}$ and $\arr{\Rules,T}$ are the same.

A \emph{higher-order rewriting system} (HRS) is a HOTRS with $T = \{ s \in \Terms(\F,\V) \mid s$
is in $\eta$-long form$\}$.

A \emph{pattern higher-order rewriting system} (PRS) is a HRS where moreover all elements of
$\Rules$ are pattern rules.

An \emph{algebraic functional system} (AFS) is a HOTRS with the following properties:
\begin{itemize}
\item $\F \supseteq \{ @_{\sigma,\tau} : (\sigma \arrtype \tau) \arrtype \sigma \arrtype \tau
  \mid \sigma,\tau \in \Types \}$;
\item $T = \{ s \in \Terms(\F,\V) \mid \forall t \subtermeq s: t$ does not have the form
  $\avar(t_1,\dots,t_n)\}$;
\item $\Rules \supseteq \Rules_\beta := \{ @_{\sigma,\tau}(\abs{\avar}{s},t) \arrz s[\avar:=t] \mid
  \sigma,\tau \in \Types \wedge s,t \in T \wedge (\avar : \sigma) \in \V \wedge s : \tau \wedge
  t : \sigma \}$.
\end{itemize}
Note that we would have the same relation if $\Rules_\beta$ were replaced by \{
$@_{\sigma,\tau}(\avar,\bvar) \arrz \avar(\bvar) \mid \sigma,\tau \in \Types \}$, but this would not
satisfy the property that rules use only terms in $T$.

\end{document}

%\section{Unconstrained first-order term rewriting}
%
%Although first-order (many-sorted) term rewriting systems can be seen as a kind of higher-order
%term rewriting system, we will present their definition separately first, and later explain how
%they can be viewed as part of the larger framework.
%
%\subsection{Terms} When considering \emph{first-order} term rewriting, we limit interest to $\F$
%with the following property: for every $(\afun : \atype) \in \F$ we have $\order(\atype) \leq 1$.
%First-order terms are those expressions $s$ such that $s : \asort$ can be derived for some
%\emph{base type} $\asort$ using the following clauses:
%\begin{itemize}
%\item if $(\afun : \atype_1 \arrtype \dots \arrtype \atype_n \arrtype \asort) \in \F$ and
%  $s_1 : \atype_1,\dots,s_n : \atype_n$ then $\afun(s_1,\dots,s_n) : \asort$;
%\item if $(\avar : \asort) \in \V$ then $\avar : \asort$.
%\end{itemize}
%We denote $\FOTerms(\F,\V)$ for the set of all first-order terms $s$.
%A first-order term of the form $\afun(s_1,\dots,s_n)$ is called a \emph{functional term} and
%$\afun$ is its root; a term $\avar$ is simply called a variable.
%If $s : \asort$ then we say that $\asort$ is the type of $s$; it is clear from the definitions
%above that each term has a unique type (which is a base type).
%
%The set $\FV(s)$ of \emph{variables} of a term $s$ is inductively defined as follows:
%\begin{itemize}
%\item $\FV(\afun(s_1,\dots,s_n)) = \FV(s_1) \cup \dots \cup \FV(s_n)$;
%\item $\FV(\avar) = \{ \avar \}$.
%\end{itemize}
%%That is, $\FV(s)$ contains all variables in $s$.
%
%\bigskip
%The \emph{subterm} relation $\subtermeq$ is defined as follows:
%\begin{itemize}
%\item $s \subtermeq s$ for all $s$;
%\item $s \subtermeq \afun(s_1,\dots,s_n)$ if $s \subtermeq s_i$ for some $i$.
%\end{itemize}
%If $s \subtermeq t$ we say that $s$ \emph{is a subterm of} $t$.
%
%\subsection{Substitution}
%
%A substitution is a function $\gamma$ that maps each variable $\avar \in \V$ to a term
%$\gamma(\avar)$ of the same type.  A substitution is applied to an arbitrary first-order term as
%follows:
%\begin{itemize}
%\item $\afun(s_1,\dots,s_n)\gamma = \afun(s_1\gamma,\dots,s_n\gamma)$;
%\item $\avar\gamma = \gamma(\avar)$.
%\end{itemize}
%
%The \emph{domain} $\domain(\gamma)$ of a substitution $\gamma$ is the set of all variables $x$
%such that $\gamma(x) \neq x$.
%We denote $[x_1:=s_1,\dots,x_n:=s_n]$ for the substitution $\gamma$ with $\gamma(x_i) = s_i$ for
%$1 \leq i \leq n$ and $\gamma(y) = y$ for $y \notin \{x_1,\dots,x_n\}$.
%For two substitutions $\gamma$ and $\delta$, we let $\gamma\delta$ denote the substitution that
%maps each variable $x$ to $\gamma(x)\delta$.
%
%\subsection{Positions}
%
%The \emph{positions} of a given first-order term are the paths to specific subterms, defined as
%follows:
%
%\begin{itemize}
%\item $\Positions(\afun(s_1,\dots,s_n)) = \{ \epsilon \} \cup \{ i \cdot p \mid 1 \leq i \leq n
%  \wedge p \in \Positions(s_i) \}$;
%\item $\Positions(\avar) = \{ \epsilon \}$.
%\end{itemize}
%Note that positions are associated to a term; thus, not every integer sequence is a position.
%
%For a term $s$ and a position $p \in \Positions(s)$, the \emph{subterm of $s$ at position $p$},
%denoted $s|_p$, is defined as follows:
%\begin{itemize}
%\item $s|_\epsilon = s$;
%\item $\afun(s_1,\dots,s_n)|_{i \cdot p} = s_i|_p$;
%\end{itemize}
%
%Note that $t \subtermeq s$ if and only if there is some position $p \in \Positions(s)$ with
%$t = s|_p$.
%If $s|_p$ has the same type as some term $t$, then $s[t]_p$ denotes $s$ with the subterm at position
%$p$ replaced by $t$.  Formally, $s[t]_p$ is obtained as follows:
%\begin{itemize}
%\item $s[t]_p = t$;
%\item $\afun(s_1,\dots,s_n)[t]_{i \cdot p} = \afun(s_1,\dots,s_{i-1},s_i[t]_p,s_{i+1},\dots,s_n)$.
%\end{itemize}
%
%\subsection{Rules and reduction}
%
%A rule is a pair $\ell \arrz r$ of two terms with the same type.
%For a given set of rules $\Rules$, the reduction relation $\arr{\Rules}$ is given by:
%\begin{itemize}
%\item if there exist $\ell \arrz r \in \Rules$ and $p \in \Positions(s)$ and substitution $\gamma$
%  such that $s|_p = \ell\gamma$, then $s \arr{\Rules} s[r\gamma]_p$.
%\end{itemize}
%
%\bigskip
%A \emph{first-order term rewriting system (TRS)} is an abstract rewriting system of the form
%$(\FOTerms(\F,\V),\arr{\Rules})$.
%
%In principle, we have defined a \emph{many-sorted} TRS here; a traditional unsorted TRS is obtained
%by limiting interest to the case $\Sorts = \{ \unitsort \}$.


